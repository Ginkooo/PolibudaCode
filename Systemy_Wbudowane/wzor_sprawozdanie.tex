\documentclass[12pt]{article}
\usepackage{graphicx}
\usepackage{listings}
\usepackage[letterpaper, margin=2cm]{geometry}
\usepackage[T1]{fontenc}
\usepackage{polski}
\usepackage[export]{adjustbox}
\usepackage[utf8]{inputenc}
\usepackage[polish]{babel}
\graphicspath{{.}}
\usepackage {tocloft}
\usepackage{hyperref}

\hypersetup{%
    colorlinks,
    citecolor=black,
    filecolor=black,
    linkcolor=black,
    urlcolor=black
}

\renewcommand{\cftsecleader}{\cftdotfill{\cftdotsep}}

\lstset{style=mystyle}

\begin{document}
    \begin{center}
        \includegraphics[width=5cm, height=5cm,]{herbPL}
        \hspace{2cm}
        \includegraphics[width=5cm, height=5cm]{herbWEII}
    \end{center}
    \centering
    {\Huge \textbf{SPRAWOZDANIE}}
    \vspace{0.5cm}
    \newline
    {\large LABORATIRIA SYSTEMÓW WBUDOWANYCH}
    \vfill
    \raggedright%
    \textbf{IMIĘ I NAZWISKO:} Piotr Czajka

    \newpage
    \textbf{NUMER ĆWICZENIA:} 3
    \newline
    \textbf{Grupa laboratoryjna:} GL02
    \newline
    \textbf{Data wykonywania ćwiczenia:} 05.04.2018
    \newline

    \newpage

    \tableofcontents

    \newpage

    \section{Zadanie 1 -- kod}% chktex 8
    \lstinputlisting{ex1.c}
    \newpage

    \section{Zadanie 1 - opis}% chktex 8
    Należało odczytać wartość temperatury z układu znajdującego się na płytce, by później ją odpowiednio przeskalować na stopnie celsjusza.
    \newpage

    \section{Zadanie 2 - kod}% chktex 8
    \lstinputlisting{ex2.c}
    \newpage

    \section{Zadanie 2 - opis}% chktex 8
    Od zadania 1 rozni sie tylko tym, ze temperaturę należało przekonwertować na stopnie Fahrenheita i Kelvina, by je następnie wyświetlić.
    \newpage

    \section{Zadanie 3 - kod}
    \lstinputlisting{ex3.c}
    \newpage

    \section{Zadanie 3 - opis}
    Należało pokazać zmianę temperatury, stosując jakiś efekt graficzny, oprócz wypisywania temperatury na ekran. Ja zastosowałem zmiany koloru czcionki w zależności od temperatury.
    \newpage

    \section{Wnioski}
    Układ nie jest w stanie sam podać prawidłowej temperatury w znanej nam jednostce. Trzeba programowo odczyt z układu przekonwertować na sensowne jednostki.

\end{document}
